% ENCODING
\usepackage[utf8]{inputenc}
\usepackage{pmboxdraw}
% \usepackage{pmboxdraw-extras}

% text alignment
\usepackage{array, ragged2e}

% URLs
\usepackage{hyperref}

\usepackage{xargs}

\usepackage{amsmath, amsfonts, amsthm, amssymb, amscd, amsxtra}

% MATH SYMBOLS
\newcommand{\F}{\mathbb{F}}
\newcommand{\N}{\mathbb{N}}
\newcommand{\Q}{\mathbb{Q}}
\newcommand{\R}{\mathbb{R}}
\newcommand{\Z}{\mathbb{Z}}
\newcommand{\C}{\mathbb{C}}
% \renewcommand{\b}{}
\def \Np {\N_{>0}}

% SET THEORY
\newcommand{\nequiv}{\not\equiv}
\renewcommand{\notin}{\not\in}
\DeclareSymbolFont{CMsymbols}{OMS}{cmsy}{m}{n}
\SetSymbolFont{CMsymbols}{bold}{OMS}{cmsy}{b}{n}
\DeclareMathSymbol{\foralll}{\mathord}{CMsymbols}{"38}
\DeclareMathSymbol{\existss}{\mathord}{CMsymbols}{"39}
\renewcommand{\forall}{\foralll\;}
\renewcommand{\exists}{\existss\;}
\DeclareRobustCommand{\set}[1]{\left\{#1\right\}}
\renewcommand{\vector}[1]{\left\langle #1 \right\rangle}
\newcommand*{\parens}[1]{\left( #1 \right)}
\newcommand{\no}{\emptyset}

% FUNCTIONS
\DeclareMathOperator{\cost}{cost}
\DeclareMathOperator{\len}{len}
\DeclareMathOperator{\rank}{rank}
\DeclareMathOperator{\sgn}{sgn}
\DeclareMathOperator{\wt}{wt}
\renewcommand{\log}[1]{\, \mathbf{log}\; #1 \,}

% ANALYSIS
\DeclareRobustCommand{\lub}[1]{\; \mathbf{l.u.b.}\; #1}
\DeclareRobustCommand{\glb}[1]{\; \mathbf{g.l.b.}\; #1}

% ALGEBRA
\newcommand{\GL}{\mathrm{GL}}
\DeclareMathOperator{\id}{id}
\DeclareMathOperator{\M}{M}
\DeclareMathOperator{\SL}{SL}
\DeclareMathOperator{\Syl}{Syl}
\DeclareMathOperator{\tr}{tr}
\DeclareMathOperator{\ev}{ev}

% COMBINATORICS

\def \dim {\mathbf{dim}}

% LOGIC
\usepackage{MnSymbol}
\def \backmodels{\leftmodels}
\DeclareRobustCommand\Iff{\;\Longleftrightarrow\;}
\DeclareRobustCommand\iff{\;\leftrightarrow\;}
\DeclareRobustCommand\lto{\rightarrow}
\DeclareRobustCommand\backimplies{\Longleftarrow}
\newcommand{\bigland}{\bigwedge}
\def \vbar{\overline{v}}
\def \sbar{\overline{s}}
\def \deduces{\vdash}
\def \isdeduced{\dashv}
\def \To{\Rightarrow}
\def \Th{\text{Th }}
\def \Mod{\text{Mod }}
\def \Cn { \text{Cn }}

% THEORY OF COMPUTATION
\DeclareRobustCommand{\shift}[1]{\; \textsc{DropFirst}\left(#1\right)}
\def \none{\sqcup}
\usepackage{tikz}
\tikzset{
  between/.style args={#1 and #2}{
        at = ($(#1)!0.5!(#2)$)
  }
}
\usetikzlibrary{automata, positioning, arrows, calc}
\tikzset{
  between base/.style args={#1 and #2}{
      between=#1.base and #2.base
  }
}

\tikzset{
  ->, % makes the edges directed
  >=stealth', % makes the arrow heads bold
  node distance=3cm, % specifies the minimum distance between two nodes. Change if necessary.
  every state/.style={thick, fill=gray!10}, % sets the properties for each ’state’ node
  initial text=$ $, % sets the text that appears on the start arrow
}

\def \EqDFA {\textsc{EQ}_{\textsc{DFA}}}
\def \sfP {\textsf{P}}
\def \sfNP {\textsf{NP}}
\def \True {\textsc{True }}
\def \False {\textsc{False }}

\DeclareRobustCommand\derives{\Rightarrow}

% Calculus
\renewcommand{\d}{\text{ d}}
\DeclareMathOperator{\erf}{\mathbf{erf}}
\DeclareMathOperator{\conv}{\mathbf{conv}}



% PROBABILITY
\newcommand{\Hd}{\texttt{\color{BrickRed}H}}
\newcommand{\Tl}{\texttt{T}}
\newcommand{\EE}{\mathop{\mathbb{E}}}
\newcommand{\PP}{\mathop{\mathbb{P}}}
\newcommand{\indic}{\mathbb{1}}
\DeclareMathOperator{\Var}{Var}

% End of proof marker
\newcommand{\qedblack}{\hfill\ensuremath{\blacksquare}}
\newcommand{\qedwhite}{\hfill\ensuremath{\square}}

% Fraktur letters
\newcommand{\frakA}{\mathfrak{A}}
\newcommand{\frakB}{\mathfrak{B}}
\newcommand{\frakC}{\mathfrak{C}}
\newcommand{\frakD}{\mathfrak{D}}
\newcommand{\frakE}{\mathfrak{E}}
\newcommand{\frakF}{\mathfrak{F}}
\newcommand{\frakG}{\mathfrak{G}}
\newcommand{\frakH}{\mathfrak{H}}
\newcommand{\frakI}{\mathfrak{I}}
\newcommand{\frakJ}{\mathfrak{J}}
\newcommand{\frakK}{\mathfrak{K}}
\newcommand{\frakL}{\mathfrak{L}} 
\newcommand{\frakM}{\mathfrak{M}}
\newcommand{\frakN}{\mathfrak{N}}
\newcommand{\frakO}{\mathfrak{O}}
\newcommand{\frakP}{\mathfrak{P}}
\newcommand{\frakQ}{\mathfrak{Q}}
\newcommand{\frakR}{\mathfrak{R}}
\newcommand{\frakS}{\mathfrak{S}}
\newcommand{\frakT}{\mathfrak{T}}
\newcommand{\frakU}{\mathfrak{U}}
\newcommand{\frakV}{\mathfrak{V}}
\newcommand{\frakW}{\mathfrak{W}}
\newcommand{\frakX}{\mathfrak{X}}
\newcommand{\frakY}{\mathfrak{Y}}
\newcommand{\frakZ}{\mathfrak{Z}}

\newcommand{\fA}{\mathfrak{A}}
\newcommand{\fB}{\mathfrak{B}}
\newcommand{\fC}{\mathfrak{C}}
\newcommand{\fD}{\mathfrak{D}}
\newcommand{\fE}{\mathfrak{E}}
\newcommand{\fF}{\mathfrak{F}}
\newcommand{\fG}{\mathfrak{G}}
\newcommand{\fH}{\mathfrak{H}}
\newcommand{\fI}{\mathfrak{I}}
\newcommand{\fJ}{\mathfrak{J}}
\newcommand{\fK}{\mathfrak{K}}
\newcommand{\fL}{\mathfrak{L}}
\newcommand{\fM}{\mathfrak{M}}
\newcommand{\fN}{\mathfrak{N}}
\newcommand{\fO}{\mathfrak{O}}
\newcommand{\fP}{\mathfrak{P}}
\newcommand{\fQ}{\mathfrak{Q}}
\newcommand{\fR}{\mathfrak{R}}
\newcommand{\fS}{\mathfrak{S}}
\newcommand{\fT}{\mathfrak{T}}
\newcommand{\fU}{\mathfrak{U}}
\newcommand{\fV}{\mathfrak{V}}
\newcommand{\fW}{\mathfrak{W}}
\newcommand{\fX}{\mathfrak{X}}
\newcommand{\fY}{\mathfrak{Y}}
\newcommand{\fZ}{\mathfrak{Z}}

% Calligraphic caps
\newcommand{\calA}{\mathcal{A}}
\newcommand{\calB}{\mathcal{B}}
\newcommand{\calC}{\mathcal{C}}
\newcommand{\calD}{\mathcal{D}}
\newcommand{\calE}{\mathcal{E}}
\newcommand{\calF}{\mathcal{F}}
\newcommand{\calI}{\mathcal{I}}
\newcommand{\calJ}{\mathcal{J}}
\newcommand{\calK}{\mathcal{K}}
\newcommand{\calL}{\mathcal{L}}
\newcommand{\calO}{\mathcal{O}}
\newcommand{\calP}{\mathcal{P}}
\newcommand{\calS}{\mathcal{S}}
\newcommand{\calT}{\mathcal{T}}
\newcommand{\calU}{\mathcal{U}}
\newcommand{\calV}{\mathcal{V}}
\newcommand{\calX}{\mathcal{X}}
\newcommand{\calY}{\mathcal{Y}}
\newcommand{\calZ}{\mathcal{Z}}

\def \cA {\calA}
\def \cB {\calB}
\def \cC {\calC}
\def \cD {\calD}
\def \cE {\calE}
\def \cF {\calF}
\def \cG {\calG}
\def \cH {\calH}
\def \cI {\calI}
\def \cJ {\calJ}
\def \cK {\calK}
\def \cL {\calL}
\def \cM {\calM}
\def \cN {\calN}
\def \cO {\calO}
\def \cP {\calP}
\def \cQ {\calQ}
\def \cR {\calR}
\def \cS {\calS}
\def \cT {\calT}
\def \cU {\calU}
\def \cV {\calV}
\def \cW {\calW}
\def \cX {\calX}
\def \cY {\calY}
\def \cZ {\calZ}

% Boldface letters
\newcommand{\ba}{\mathbf{a}}
\newcommand{\bb}{\mathbf{b}}
\newcommand{\bc}{\mathbf{c}}
\newcommand{\bd}{\mathbf{d}}
\newcommand{\be}{\mathbf{e}}
% \newcommand{\bf}{\mathbf{f}}
\newcommand{\bp}{\mathbf{p}}
\newcommand{\bq}{\mathbf{q}}
\newcommand{\br}{\mathbf{r}}
\newcommand{\bu}{\mathbf{u}}
\newcommand{\bv}{\mathbf{v}}
\newcommand{\bx}{\mathbf{x}}
\newcommand{\by}{\mathbf{y}}
\newcommand{\bz}{\mathbf{z}}
\newcommand{\tbp}{\mathbf{\widetilde{p}}}

% BOLD CAPITALS
\newcommand{\bA}{\mathbf{A}}
\newcommand{\bB}{\mathbf{B}}
\newcommand{\bC}{\mathbf{C}}
\newcommand{\bD}{\mathbf{D}}
\newcommand{\bE}{\mathbf{E}}
\newcommand{\bF}{\mathbf{F}}
\newcommand{\bG}{\mathbf{G}}
\newcommand{\bH}{\mathbf{H}}
\newcommand{\bI}{\mathbf{I}}
\newcommand{\bJ}{\mathbf{J}}
\newcommand{\bK}{\mathbf{K}}
\newcommand{\bL}{\mathbf{L}}
\newcommand{\bM}{\mathbf{M}}
\newcommand{\bN}{\mathbf{N}}
\newcommand{\bO}{\mathbf{O}}
\newcommand{\bP}{\mathbf{P}}
\newcommand{\bQ}{\mathbf{Q}}
\newcommand{\bR}{\mathbf{R}}
\newcommand{\bS}{\mathbf{S}}
\newcommand{\bT}{\mathbf{T}}
\newcommand{\bU}{\mathbf{U}}
\newcommand{\bV}{\mathbf{V}}
\newcommand{\bW}{\mathbf{W}}
\newcommand{\bX}{\mathbf{X}}
\newcommand{\bY}{\mathbf{Y}}
\newcommand{\bZ}{\mathbf{Z}}

% Special math symbols
\newcommand{\eps}{\varepsilon}
\renewcommand*{\epsilon}{\varepsilon}
\renewcommand*{\phi}{\varphi}
\newcommand{\ceq}{\subseteq}
\newcommand{\ang}[1]{\langle{} #1 \rangle}
\newcommand{\ceil}[1]{\lceil{} #1 \rceil}
\newcommand{\floor}[1]{\lfloor{} #1 \rfloor}

% bold symbols
% USE MODERN FONTS!
% \usepackage{anyfontsize}
% \renewcommand{\b}[1]{\boldsymbol{#1}}

% Problem names and other small-caps constants
\newcommand{\inv}{\textsc{inv}\xspace}

% Useful for marking steps of a derivation to explain later
\newcommand{\circled}[1]{\raisebox{.5pt}{\textcircled{\raisebox{-.1pt}{\scriptsize #1}}}}


% Page size and margins
% \usepackage[left=1in,right=1in,top=1.3in,bottom=1.3in,nofoot]{geometry}
\usepackage{fancyhdr}   % for fancy header
\usepackage{fancyvrb}   % for fancy verbatim
\usepackage{graphicx}   % for including images
\usepackage{enumerate}  % for enumerating lists
\usepackage{enumitem}
% \usepackage[dvipsnames]{xcolor}
\usepackage{tcolorbox}

% Custom colors
\definecolor{crimson}{rgb}{0.86, 0.08, 0.24}
\definecolor{teal}{rgb}{0.0, 0.5, 0.5}
\definecolor{zaffre}{rgb}{0.0, 0.08, 0.66}
\definecolor{DarkOliveGreen}{rgb}{0.33, 0.42, 0.18}
\definecolor{green}{rgb}{0.0, 0.5, 0.0}
\definecolor{grey}{rgb}{0.5, 0.5, 0.5}
\definecolor{electriclime}{rgb}{0.8, 1.0, 0.0}
\definecolor{glaucous}{rgb}{0.38, 0.51, 0.71}
\newcommand{\crim}{\textcolor{crimson}}
\newcommand{\teal}{\textcolor{teal}}
\newcommand{\zaff}{\textcolor{zaffre}}
\newcommand{\black}{\textcolor{black}}
\newcommand{\darkgreen}{\textcolor{DarkOliveGreen}}
\newcommand{\green}{\textcolor{green}}
\newcommand{\grey}{\textcolor{grey}}

% Block coloring
\newenvironment{blockcolor}{\par \color{crimson} {\par}}

% -- Defining colors:
\definecolor{codegreen}{rgb}{0,0.6,0}
\definecolor{codegray}{rgb}{0.5,0.5,0.5}
\definecolor{codepurple}{rgb}{0.58,0,0.82}
\definecolor{backcolour}{rgb}{0.95,0.95,0.92}
\definecolor{dkgreen}{rgb}{0,0.6,0}
\definecolor{gray}{rgb}{0.5,0.5,0.5}
\definecolor{mauve}{rgb}{0.58,0,0.82}

\usepackage{multicol}

% Answer BOX
\usepackage{microtype}
\usepackage{mdframed}
\newmdenv[%
  skipabove=6pt,
  skipbelow=6pt,
  innertopmargin=6pt,
  leftmargin=-5pt,
  rightmargin=-5pt, 
  innerleftmargin=5pt,
  innerrightmargin=5pt,
  backgroundcolor=black!10
]{Answer}%


% Header BOX
\newcommand{\handout}[6]{
  \noindent
  \begin{center}
  \setlength{\fboxrule}{1.2pt}
  \framebox{
    \vbox{
      \hbox to 5.78in { \textbf{#6} \hfill {\bf #2} }
      \vspace{4mm}
      \hbox to 5.78in { {\Large \hfill {\textbf{ #5 }}  \hfill} }
      \vspace{2mm}
      \hbox to 5.78in { {\textit{\textbf{#3 \hfill #4}}} }
    }
  }
  \setlength{\fboxrule}{0.2pt}
  \end{center}
  \vspace*{4mm}
}

% Header BOX
\newcommand{\PSET}[5]{\handout{#1}{#2}{Prof.\ #3}{Student: #4}{PSET #1}{#5}}

\newcommand{\homework}[5]{\handout{#1}{#2}{Prof.\ #3}{Student: #4}{Homework assigned #1}{#5}}

\newcommand{\exam}[5]{\handout{#1}{#2}{Prof.\ #3}{Student: #4}{Exam #1}{#5}}

\newcommand{\midterm}[5]{\handout{#1}{#2}{Prof.\ #3}{Student: #4}{Mid-Term Exam #1}{#5}}

\newcommand{\reading}[5]{\handout{#1}{#2}{Prof.\ #3}{Student: #4}{Reading assigned #1}{#5}}

\newcommand{\quiz}[5]{\handout{#1}{#2}{Prof.\ #3}{Student: #4}{Quiz #1}{#5}}

\newcommand{\statement}[5]{\handout{#1}{#2}{Prof.\ #3}{Student: #4}{Participation Statement #1}{#5}}

\newcommand{\final}[5]{\handout{#1}{#2}{Prof.\ #3}{Student: #4}{FINAL EXAM --- #1}{#5}}
% Credit Statement
\newcommand{\CreditStatement}[1]{
  \noindent
  \begin{center} {
    \bf Credit Statement
  }
  \end{center}
  { #1 }
}

% Problem Counter
\newenvironment{problem}[1][]%
{%
\ifx&#1&%
  % #1 is empty
  \stepcounter{problem}
\else
  % #1 is nonempty
  \setcounter{problem}{#1}
\fi
\setcounter{section}{\value{problem}}
\setcounter{equation}{0}
\vspace{.2cm} \noindent {\center  \textbf{\\ Problem \arabic{problem}.\\}} {}~%
}{%
% \vspace{.2cm}%
}
% Package Imports
% \usepackage{amssymb,amsthm,amsmath,amstext}
\usepackage{mathdots} % for \dots
  % \dotsc -- dots with commas.
  % \dotsb -- dots with binary operators.
  % \dotsm -- multiplication dots.
  % \dotsi -- dots with integrals.
  % \dotso -- "other dots".
\usepackage{wasysym, stackengine, makebox, tikz-cd}
\newcommand\isom{\mathrel{\stackon[-0.1ex]{\makebox*{\scalebox{1.08}{\AC}}{=\hfill\llap{=}}}{{\AC}}}}
\newcommand\nvisom{\rotatebox[origin=cc] {-90}{$ \isom $}}
\newcommand\visom{\rotatebox[origin=cc] {90} {$ \isom $}}




% \newcommand{\id}{\mathbf{id}\;}

% matrices -- vertical separators
\makeatletter
\renewcommand*\env@matrix[1][*\c@MaxMatrixCols{ c}]{%
  \hskip -\arraycolsep{}
  \let\@ifnextchar\new@ifnextchar{}
  \array{#1}}
\makeatother

% \usepackage{accode}
\usepackage{tikz}

% long multiplications
\usepackage{xlop}

% custom functions.
\renewcommand{\gcd}[2]{\mathbf{gcd}\;(#1,\;#2)}
\newcommand{\lcm}[2]{\mathbf{lcm}\;(#1,\;#2)}
\newcommand{\Therefore}{\therefore\;}
\newcommand{\However}{\dot{}\hspace{.045in}.\hspace{.045in} \dot{}\hspace{.095in}}

% resume includes
\usepackage[utf8]{inputenc}
\usepackage{textcomp}
\usepackage{CJKutf8}
\usepackage[lf]{ebgaramond}

\usepackage[OT1]{fontenc}
\usepackage{enumitem}
\usepackage[scale=.75]{geometry}
\usepackage{url}

\pagestyle{headings}

\setlength\parindent{2em}

\thispagestyle{empty}

\newcommand{\cvsubsection}[1]{\subsection*{\hspace{1.45em}#1}}

% import := definition
\usepackage{colonequals}

\usepackage{listings}

\lstset{frame=tb,
  backgroundcolor=\color{backcolour},   
  commentstyle=\color{codepurple},
  keywordstyle=\color{NavyBlue},
  numberstyle=\tiny\color{codegray},
  stringstyle=\color{codepurple},
  basicstyle=\ttfamily\footnotesize\bfseries,
  breakatwhitespace=false,         
  breaklines=true,                 
  captionpos=t,                    
  keepspaces=true,                 
  numbers=left,                    
  numbersep=5pt,                  
  showspaces=false,                
  showstringspaces=false,
  showtabs=false,                  
  tabsize=2,
  % escapeinside={\%*}{*)},          % if you want to add LaTeX within your code
}

\usepackage{booktabs}


\DeclareMathOperator{\Aut}{Aut}
\DeclareMathOperator{\opspan}{span}
\DeclareMathOperator{\Tr}{Tr}
\DeclareMathOperator{\Frac}{Frac}
\DeclareMathOperator{\ord}{ord}
\DeclareMathOperator{\Sym}{Sym}

% \usepackage{theorem}
% \theorembodyfont{\rm}%\normalfont
\numberwithin{equation}{section}
\newtheorem{theorem}[equation]{Theorem}
\newtheorem{lemma}[equation]{Lemma}
\newtheorem{proposition}[equation]{Proposition}
\newtheorem{corollary}[equation]{Corollary}
\newtheorem{claim}[equation]{Claim}
\newtheorem{observation}[equation]{Observation}

\theoremstyle{definition}
\newtheorem{definition}[equation]{Definition}
\newtheorem{defn}[equation]{Definition}
\newtheorem{example}[equation]{Example}
\newtheorem{xca}[equation]{Exercise}
\newtheorem{notation}[equation]{Notation}
\theoremstyle{remark}
\newtheorem{remark}[equation]{Remark}
\numberwithin{equation}{section}

\newenvironment{proof*}[1][\proofname]{%
  \begin{proof}[#1]$ $\par\nobreak\ignorespaces
}{%
  \end{proof}
}


\usepackage[normalem]{ulem}
\usepackage{fullpage}
\usepackage{colonequals}
\usepackage{mathtools}
\usepackage{mathrsfs}


\usepackage{hyperref}
\hypersetup{colorlinks=true,urlcolor=blue,citecolor=blue,linkcolor=blue}

\numberwithin{equation}{section}

% \usepackage{amssymb}
% \usepackage{amsthm}
% \usepackage{amsmath}
% \usepackage{amsxtra}


\setlength{\hfuzz}{4pt}

% \DeclarePairedDelimiter{\abs}{\lvert}{\rvert}
\newcommand{\abs}[1]{\left\vert #1 \right\vert}

\newcommand{\defi}[1]{\textsf{#1}} % for defined terms

\renewcommand{\baselinestretch}{1.5} 

\usepackage{titling}
\usepackage[english]{babel}
\usepackage[utf8]{inputenc}
\usepackage{graphicx}
\usepackage[colorinlistoftodos]{todonotes}
\usepackage{subfig}
% \usepackage{mdframed} 
\usepackage{color}
\usepackage{tabu}
\usepackage{tikz}
\usepackage{enumerate}
\usepackage{multicol}
\usepackage{pgfplots}
\usepackage{csquotes}
\pgfplotsset{compat=1.18}
% \usepackage[style=iso]{datetime2}
\usepackage[mmddyyyy]{datetime}
\usepackage{datenumber}
\usepackage{multirow}
\usepackage{float}                    % prevent table repositioning.
\usepackage{lipsum}


% SQL Symbols
\def\ojoin{\setbox0=\hbox{$\bowtie$}%
  \rule[-.02ex]{.25em}{.4pt}\llap{\rule[\ht0]{.25em}{.4pt}}}
\def\leftouterjoin{\mathbin{\ojoin\mkern-5.8mu\bowtie}}
\def\rightouterjoin{\mathbin{\bowtie\mkern-5.8mu\ojoin}}
\def\fullouterjoin{\mathbin{\ojoin\mkern-5.8mu\bowtie\mkern-5.8mu\ojoin}}


\renewcommand{\theenumi}{\alph{enumi}}
\usepackage{enumitem}

% enumalph
\newenvironment{enumalph}{
  \begin{enumerate}[label=(\alph*)]
}{\end{enumerate}}

% enumroman
\newenvironment{enumroman}{
  \begin{enumerate}[label=(\roman*)]
}{\end{enumerate}}

% enumarabic
\newenvironment{enumarabic}{
  \begin{enumerate}[label=\textbf{\arabic*.}]
}{\end{enumerate}}

\newenvironment{enumarabic*}{
  \begin{enumerate}[label*=\textbf{\arabic*.}]
}{\end{enumerate}}

\usepackage[english]{cleveref}

\DeclareRobustCommand{\step}{\bigskip\noindent}


\newcounter{problem}
\setcounter{problem}{0}

\usepackage{bookmark}

% \newcommand{\floatfootnote}[1]{\ifx\[$\else\footnote{#1}\fi}

\usepackage{stmaryrd}

\DeclareRobustCommand{\dabs}[1]{\llbracket #1 \rrbracket}

% \usepackage{mtpro2}

\def \max {\mathbf{max}}
\def \min {\mathbf{min}}

\usepackage{cases}  % for numcases
