\def \ransac { \textsc{RANSAC} }
\def \cnn { \textsc{CNN} }
\section{Existing Methods}
The most common approaches to waterline detection are:
\subsection{ Use of Hough transforms with \ransac}
This is an edge-based method for waterline detection.
A raw test image is resized and fed through a \cnn, producing a a segmentation mask
that follows the contour of the boundaries between the water and any obstacles
and the sky.

\begin{figure}[H]
  \includegraphics*[width=0.8\textwidth,height=50mm]{one-prelim.png}
  \caption{
    Semantic segmentation example~\cite{zhan-2020}.
  }
\end{figure}

To effectively extract the waterline, post-processing is done as follows:
\begin{enumroman}
  \item First, a median filter is applied to the segmentation mask,
    reducing narrow protrusions such as sailboat masts.
  \item Binary edge detection is then applied to isolate the pixels on the boundary
    between the water and the sky.
  \item A probabilistic Hough transform is applied to the edge-detected image,
    producing a set of possible lines and further cutting out shorter lines.
    Only the pixels composing a long edge are retained.
  \item Finally, linear regression is performed using \ransac to reduce the
    influence of sharp edges on the contour.
\end{enumroman}

\begin{figure}[htb]
  \includegraphics*[width=0.8\textwidth]{one.png}
  \caption{
    Hough transform + \ransac method for waterline detection
    \cite{steccanella-2019,zhan-2017}.
  }
\end{figure}

\begin{figure}[H]
  \includegraphics*[width=0.7\textwidth]{one-results.png}
  \caption{
    Experimental results presented by authors
    \cite{steccanella-2019}.
  }
\end{figure}

\step
\subsubsection{Disadvantages}
As with any edge-based method, two key challenges must be overcome\cite{zhan-2017}.
\begin{enumroman}
  \item Minimization of false negatives (noise).
  \item Efficient and robust identification of the waterline from the  noise.
\end{enumroman}

\subsection{Regional-based Approaches}
Regional-based waterline detection algorithms seek to exploit regional features
such as color and texture to more effectively isolate the waterline.
One of the most common approaches works as follows:
\begin{enumroman}
  \item First, detect the sky region using some of its properties
    with the assumption being that the waterline separates the sky
    from the water. Features used to detect the sky can include
    position, color, texture, and row gradient.
  \item Pick out the pixels that are likely to be waterline pixels
    by using the sky region as a reference.
    This can be done by using the sky region as a mask and
    applying a threshold to the color or texture of the pixels
    in the sky region.
  \item Finally, use \ransac to fit a line to the pixels
\end{enumroman}

\subsubsection{Disadvantages}
In some cases, the sky region does not border the water in some
parts of the image. In those cases this method does not work well.
To counter this, Zarifar et al.~\cite{zarifar-2008} propose
a method that combines the regional-based approach with
an edge-based approach, and picks the waterline based on
the results of both methods.

\begin{figure}[H]
  \includegraphics*[width=\textwidth]{two.png}
  \caption{
    Examples of individual and combined results using
    region-based and edge-based methods~\cite{zarifar-2008}.
  }
\end{figure}

While the combined method works better, it also results in computational
redundancies, since multiple, different methods have to be run on the same image.
